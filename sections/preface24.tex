

\hspace{2em} 本学期实验指导书保留上一年丰富详实的内容外,进一步完善课程实验设置、平衡各项实验难度。修订版实验指导书通过如下github地址写作和发布:\url{https://github.com/shiqi-dai/Distributed-Machine-Learning-Experiment-Document-24-Spring}

\hspace{2em} 实验内容所做修改如下:

\hspace{2em} 由于同学对学习使用框架Mindspore的积极性不高的问题,我们在难度最低的实验一:梯度下降单机优化中要求同时采用Pytorch和MindSpore框架写优化器类,旨在调动大家查阅说明文档学习新框架的能力。同时将实现的优化器类改为SGDM,Adam两个在CV、NLP、RL、语音合成等领域的优化方法,贴近平日科研任务所用。

\hspace{2em} 对于实验报告撰写存在的各式各样的问题,我们在实验指导书新增一节实验报告撰写要求,并且明确在实验三:数据并行的实验报告要求中增加报告内容,确保同学们更准确明白掌握实验内容。

\hspace{2em} 本课程热情欢迎各位同学共同构建课程知识库,可以将自己对实验指导书的建议另附在实验报告中,您对实验指导书的贡献将会被酌情考虑,可能会影响最终的评分。

\vspace{5em}
\rightline{助教代诗琦\ \ 于2024年2月}



